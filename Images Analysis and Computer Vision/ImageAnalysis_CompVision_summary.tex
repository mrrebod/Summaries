%
% Zusammenfassung Image Analysis & Computer Vision D-ITET
% ===========================================================================
% Author:			Marco Dober
% Version:			0.1
% Last changed: 	19.09.2019	
% ---------------------------------------------------------------------------

\documentclass[a4paper, fontsize=8pt, landscape, DIV=1]{scrartcl}
\usepackage{lastpage}
\usepackage{hyperref}
\usepackage[graphicx]{realboxes}
% Include general settings and customized commands
\input{settings/general}
\input{settings/commands}
%change page style for header
\pagestyle{fancy}
\footskip 20pt
\rhead{Marco Dober}
\lhead{Image Analysis \& Computer Vision}
\chead{\thepage}
\cfoot{}
\headheight 17pt \headsep 10pt
\title{Image Analysis \& Computer Vision}
\author{Marco Dober}
\date{\today}


\begin{document}
	\setcounter{secnumdepth}{3} %no enumeration of sections
	\begin{multicols*}{4}
		%
		\section*{Disclaimer}
		This summary is part of the lecture ``ETH Image Analysis \& Computer Vision'' by Prof. Van Gool, Prof. Konukoglu and Prof. Goksel  (HS19). It is based on the lecture slides and script. \\[6pt]
		Please report errors to \href{mailto:doberm@student.ethz.ch}{doberm@student.ethz.ch} such that others can benefit as well.\\[6pt]	
		The upstream repository can be found at \href{https://github.com/mrrebod/Summaries}{https://github.com/mrrebod/Summaries}
		\vfill\null
		\pagebreak
		
		\maketitle 
		\thispagestyle{fancy}
		
		\section{Introduction}
		Vision is important:
		\begin{itemize}[noitemsep, label={$\blacktriangleright$}]
			\item Half our brain is devoted to it
			\item Developed many times douring evolution
			\item It is non-contact
			\item It can be implemenmted with high-resolution
			\item Works with ambient EM-waves
			\item yields color, texture, depth, motion, shape
		\end{itemize}
		\textcolor{red}{Take home message}:\\
		\textbf{\textcolor{red}{For people vision is their most crucial sense, for good reason}}
		\subsection{Perception of vision}
		\begin{minipage}[t]{0.49\columnwidth}
			\begin{flushleft}
				{\centering Perception of \textbf{intensity}\\}
				\includegraphics[width=\columnwidth, height=2cm]{images/Introduction/perc_intensity.png} 
				The gray fields have the same intensity (same gray tone).
			\end{flushleft}
		\end{minipage}
		\begin{minipage}[t]{0.49\columnwidth}
			\begin{flushright}
				{\centering Perception of \textbf{color}\\}
				\includegraphics[width=\columnwidth, height=2cm]{images/Introduction/perc_color.png}
				The red squares have equal color
			\end{flushright}
		\end{minipage}
		\hrule
		\begin{minipage}[t]{0.49\columnwidth}
			\begin{flushleft}
				{\centering Perception of \textbf{length}\\}
				\includegraphics[width=\columnwidth, height=2cm]{images/Introduction/perc_length.png} 
				the horizontal lines are equally long.
			\end{flushleft}
		\end{minipage}
		\begin{minipage}[t]{0.49\columnwidth}
			\begin{flushright}
				{\centering \textbf{Lines being straight}
				\includegraphics[width=2cm, height=2cm]{images/Introduction/perc_straight.png}\\}
				The lines do not have any curvature
			\end{flushright}
		\end{minipage}	
		\hrule
		\begin{minipage}[t]{0.49\columnwidth}
			\begin{flushleft}
				{\centering Perception of \textbf{parallelism}\\}
				\includegraphics[width=\columnwidth, height=2cm]{images/Introduction/perc_parallel.png} 
				All lines are parallel.
			\end{flushleft}
		\end{minipage}
		\begin{minipage}[t]{0.49\columnwidth}
			\begin{flushright}
				{\centering Perception of \textbf{curvatures}
				\includegraphics[width=2cm, height=2cm]{images/Introduction/perc_curvature.png}\\}
				There is no spiral.
			\end{flushright}
		\end{minipage}
		\begin{minipage}[t]{0.49\columnwidth}
			\begin{flushleft}
				{\centering Perception of \textbf{motion}\\}
				\includegraphics[width=\columnwidth, height=3cm]{images/Introduction/perc_motion.png} 
				The pole rotates about the vertical, it does not translate vertically.
			\end{flushleft}
		\end{minipage}
		\begin{minipage}[t]{0.49\columnwidth}
		\begin{flushright}
				{\centering The role of \textbf{context}
				\includegraphics[width=3cm, height=3cm]{images/Introduction/perc_context.png}\\}
				All encircled patterns are identical!
			\end{flushright}
		\end{minipage}
		\par 
		\textcolor{red}{Take home message}:\\
		\textbf{\textcolor{red}{Effective vision needs more than sheer filtering and measuring}}
		\subsection{Applications}
		Most early applications where found in \textbf{production environments}, as these allow for \textbf{controlled conditions} and have \textbf{little uncertainty}. But some areas do not allow for much control: medical IP, remote sensing, surveillance, etc.\\
		Currently Computer Vision (CV) is conquering the less controllable areas by storm:
		\par  
		\begin{minipage}[t]{0.49\columnwidth}
			\begin{flushleft}
		 		{\centering \textbf{Image enhancement:} mobile $\rightarrow$ DSLR\\}
		 		\includegraphics[width=\columnwidth, height=5cm]{images/Introduction/app_enhancement.png} 
			\end{flushleft}
		\end{minipage}
		\begin{minipage}[t]{0.49\columnwidth}
			\begin{flushright}
				{\centering \textbf{Image retrieval, captioning}\\}
				\includegraphics[width=\columnwidth, height=5cm]{images/Introduction/app_retrieval.png}
			\end{flushright}
		\end{minipage}
		\hrule
		\begin{minipage}[t]{0.49\columnwidth}
			\begin{flushleft}
				{\centering \textbf{Autonomous vehicles}\\}
				\includegraphics[width=\columnwidth, height=3cm]{images/Introduction/app_autonomous_vehicles.png} 
			\end{flushleft}
		\end{minipage}
		\begin{minipage}[t]{0.49\columnwidth}
			\begin{flushright}
				{\centering \textbf{Visual surveillance}\\}
				\includegraphics[width=\columnwidth, height=3cm]{images/Introduction/app_surveillance.png}
			\end{flushright}
		\end{minipage}
		\hrule
		\begin{minipage}[t]{0.49\columnwidth}
			\begin{flushleft}
				{\centering \textbf{Augmented Reality, e.g sports}\\}
				\includegraphics[width=\columnwidth, height=2.5cm]{images/Introduction/app_augm_reality.png} 
			\end{flushleft}
		\end{minipage}
		\begin{minipage}[t]{0.49\columnwidth}
			\begin{flushright}
				{\centering \textbf{Computer-assisted surgery}\\}
				\includegraphics[width=\columnwidth, height=2.5cm]{images/Introduction/app_surgery.png}
			\end{flushright}
		\end{minipage}			
		\par 
		\textcolor{red}{Take home message}:\\
		\textbf{\textcolor{red}{It is feasible now to let most things see and interpret their environment.}}
		
		\subsection{The nature of light}
		There are three different models to describe optical systems: 
		\begin{enumerate}[noitemsep]
			\item Geometrical optics
			\item Physical optics $\rightarrow$ \textcolor{red}{\textbf{wave character}}   
			\item Quantum-mechanical optics
		\end{enumerate}
		In this course we mainly look at physical optics and the wave character of light. 		
		\subsubsection{Light as EM-waves}
		Self-sustaining exchange of electric and magnetic fields. An EM-wave is characterized with the following properties: 
		\includegraphics[width=\columnwidth]{images/Introduction/em_wave.png}			
			\begin{itemize}[noitemsep]
				\item \textbf{Wavelength}
				\item Direction of \textbf{propagation}
				\item \textbf{Amplitude} of E
				\item \textbf{Phase} 
				\item Direction of \textbf{polarization }
			\end{itemize}
		\textbf{The spectrum:}\\
		Normal ambient light is a mixture of wavelengths, polarization directions and phases. The visible range for humans is only a small fraction of the EM-waves-spectrum.\\
		\begin{center}
			\begin{tabular}{c l l}
				\hline 
				\hline
				Wavelength [$nm$] &  & Color \\ 
				\hline 
				380 - 450 & $\rightarrow$ & \textcolor{violet}{violet} \\ 
				450 - 490 & $\rightarrow$ & \textcolor{blue}{blue} \\ 
				490 - 560 & $\rightarrow$ & \textcolor{green}{green} \\ 
				560 - 590 & $\rightarrow$ & \textcolor{yellow}{yellow} \\ 
				590 - 630 & $\rightarrow$ & \textcolor{orange}{orange} \\ 
				630 - 760 & $\rightarrow$ & \textcolor{red}{red} \\
				\hline
				\hline 
			\end{tabular}
		\end{center}
		%\includegraphics[height=3cm]{images/Introduction/visible_spectrum.png}
		The visible range differs from humans to animals and also cameras may have different spectral sensitivities. There are also cameras for non-visible light such as infrared. The following picture shows the three color cones humans have and their sensitivity range: 
		\includegraphics[width=\columnwidth]{images/Introduction/human_cones.png}
		  
		\subsubsection{Interactions with matter}
		We look at the following types of interaction with matter: 
		\begin{enumerate}[noitemsep]
			\item \textbf{Absorption}
				\begin{itemize}[label={$\rightarrow$}]
					\item blue water
				\end{itemize}
			\item \textbf{Scattering}
				\begin{itemize}[label={$\rightarrow$}]
					\item blue sky
					\item red sunset
				\end{itemize}
			\item \textbf{Reflection}
				\begin{itemize}[label={$\rightarrow$}]
					\item colored ink
				\end{itemize} 
			\item \textbf{Refraction}
				\begin{itemize}[label={$\rightarrow$}]
					\item dispersion by a prism
				\end{itemize} 
			\item \textbf{Diffraction} 
		\end{enumerate}
		 We look at few of those in more detail:
		 \par
		  
		 \textbf{1. Absorption}\\
		 A nice example of absorption is earth's atmosphere which absorbs certain wavelengths of the incoming light. The absorbed frequencies correspond to resonance frequencies of molecules in earth's atmosphere.\\ 
		 \includegraphics[width=\columnwidth]{images/Introduction/absorption_earth.png}
		 \clearpage 
		 
		 \textbf{2. Scattering}\\
		 There are three types of scattering depending on the relative sizes of particles and wavelengths:
		 \begin{enumerate}[label=(\alph*)]
		 	\item Small particles: \textcolor{red}{\textbf{Rayleigh}} (strong wavelength dependent)
		 	\item Comparable size: \textcolor{red}{\textbf{Mie}} (weakly wavelength dependent)
		 	\item Large particles: \textcolor{red}{\textbf{Non-selective}} (wavelength independent)
		 \end{enumerate}
	 	If we look at the scattered energy it looks as follows:\\ 
	 	\includegraphics[width=\columnwidth]{images/Introduction/scattered_energy.png}
	 	
	 	Let's see some examples of these different scatter-types in our atmosphere:\\ 
	 	\begin{minipage}[t]{0.49\columnwidth}
	 		\begin{flushleft}
	 			\includegraphics[width=\columnwidth]{images/Introduction/ex_ray_non.png}\\
	 		\end{flushleft}
	 	\end{minipage}
	 	\begin{minipage}[b]{0.49\columnwidth}
	 		\begin{flushleft}
	 			\textbf{Rayleigh}: Tyndall effect (blue sky, red setting sun)\\
	 			\textbf{Non-selective}: Grey clouds
	 		\end{flushleft}
	 	\end{minipage}
 		%\hrule
 		\par 
 		\begin{minipage}[t]{0.49\columnwidth}
 			\begin{flushleft}
 				\includegraphics[width=\columnwidth]{images/Introduction/ex_mie.png}\\
 			\end{flushleft}
 		\end{minipage}
 		\begin{minipage}[b]{0.49\columnwidth}
 			\begin{flushleft}
 				\textbf{Mie}: Colored cloud from volcanic eruption
 				\vspace{0.6cm}
 			\end{flushleft}
 		\end{minipage}
		\par 
		\textbf{3. Reflection:}\\
		In mirror reflection we have:\\ angle of reflection = angle of incident.\\
		Two different categories of reflective materials:\\
		\begin{minipage}[t]{0.49\columnwidth}
			\begin{flushleft}
				\includegraphics[width=\columnwidth]{images/Introduction/refl_diel.png}\\
			\end{flushleft}
		\end{minipage}
		\begin{minipage}[b]{0.49\columnwidth}
			\begin{flushleft}
				\textbf{Dielectric:}\\
				For parallel polarization there exists the Brewster angle where $r=0$.
				\vspace{1.2cm}
			\end{flushleft}
		\end{minipage}
		\par 
		\begin{minipage}[t]{0.49\columnwidth}
			\begin{flushleft}
				\includegraphics[width=\columnwidth]{images/Introduction/refl_metal.png}\\
			\end{flushleft}
		\end{minipage}
		\begin{minipage}[b]{0.49\columnwidth}
			\begin{flushleft}
				\textbf{Conductor:}\\
				Strong reflectors under all angles, more or less preserve polarization.
				\vspace{1.2cm}
			\end{flushleft}
		\end{minipage}
	
		We differentiate three types of reflection which depend on the surface structure: 
		\begin{minipage}[t]{0.49\columnwidth}
			\begin{flushleft}
				\includegraphics[width=\columnwidth]{images/Introduction/refl_diffuse.png}\\
			\end{flushleft}
		\end{minipage}
		\begin{minipage}[b]{0.49\columnwidth}
			\begin{flushleft}
				\textbf{Diffuse:}\\
				Also called \textbf{Lambertian}. Rough surfaces.
				\vspace{0.2cm}
			\end{flushleft}
		\end{minipage}
		\par
		\begin{minipage}[t]{0.49\columnwidth}
			\begin{flushleft}
				\includegraphics[width=\columnwidth]{images/Introduction/refl_mirror.png}\\
			\end{flushleft}
		\end{minipage}
		\begin{minipage}[b]{0.49\columnwidth}
			\begin{flushleft}
				\textbf{Specular:}\\
				Mirror-like surfaces.
				\vspace{0.5cm}
			\end{flushleft}
		\end{minipage}
		\par 
		\begin{minipage}[t]{0.49\columnwidth}
			\begin{flushleft}
				\includegraphics[width=\columnwidth]{images/Introduction/refl_mixed.png}\\
			\end{flushleft}
		\end{minipage}
		\begin{minipage}[b]{0.49\columnwidth}
			\begin{flushleft}
				\textbf{Mixed:}\\
				Mix of diffuse and specular.
				\vspace{0.5cm}
			\end{flushleft}
		\end{minipage}
		\par 
		\textbf{4. Refraction:}\\
		\begin{minipage}[t]{0.39\columnwidth}
			\begin{flushleft}
				\includegraphics[width=\columnwidth]{images/Introduction/snells_law.png}\\
			\end{flushleft}
		\end{minipage}
		\begin{minipage}[b]{0.59\columnwidth}
			\begin{flushleft}
				Effect of the bending of light if it hits an interface of two materials with different refraction index $n=\sqrt{\epsilon\mu}$. The bending is described through \textbf{\textcolor{red}{Snell's law}}:\\
				%\vspace{0.2cm}
				\ceqbox{n_1\sin\theta_1=n_2\sin\theta_2}
				\vspace{1.1cm}
			\end{flushleft}
		\end{minipage}
		\par 
		\begin{minipage}[t]{0.49\columnwidth}
			\begin{flushleft}
				\includegraphics[width=\columnwidth]{images/Introduction/dispersion.png}\\
			\end{flushleft}
		\end{minipage}
		\begin{minipage}[b]{0.49\columnwidth}
			\begin{flushleft}
				\textbf{Dispersion:}\\
				The bending is dependent of the frequency (wavelength) of the light. 
				\vspace{0.2cm}
			\end{flushleft}
		\end{minipage}
		\vfill\null
		\columnbreak

		\section{Image Acquisition}	
		\subsection{Illumination}
		Well designed illumination often is key in visual inspection and can extremely simplify the image processing. Here is an overview of different illumination techniques:
		
		\subsubsection{Back-lighting}
		\begin{center}
			\includegraphics[width=0.7\columnwidth,]{images/ImageAcq/back_lighting.png}\\
		\end{center}
		
		\textbf{How:}\\
		Lamps are placed behind a transmitting diffuser plate which is located behind the object
		\par 
		\textbf{Why:}\\
		Creates \textbf{high-contrast} silhouette images, easy to handle with \textbf{binary vision} (only two intensity levels, black and white). Often used in inspection.
	
		\subsubsection{Directional-lighting}
		\begin{center}
			\includegraphics[width=0.7\columnwidth,]{images/ImageAcq/directional_lighting.png}\\
		\end{center}
	
		\textbf{How:}\\
		Light source shines directly on object, maybe under a certain angle.
		\par 
		\textbf{Why:}\\
		\vspace{-0.5cm}
		\begin{itemize}[noitemsep]
			\item Generation of \textbf{specular reflection} (e.g. crack detection (see figure above))
			\item Generation of \textbf{sharp shadows}
		\end{itemize}
		\vfill\null
		\columnbreak
		
		\subsubsection{Diffuse-lighting}
		\begin{minipage}[t]{0.49\columnwidth}
			\begin{flushleft}
				\includegraphics[width=\columnwidth]{images//ImageAcq/diffusive_lighting.png}\\
			\end{flushleft}
		\end{minipage}
		\begin{minipage}[b]{0.49\columnwidth}
			\begin{flushleft}
				\textbf{Left:}\\
				Direct lighting produced large changes in brightness due to specular reflection.\\
				\vspace{0.2cm}
				\textbf{right:}\\
				Diffusive lighting reduces bright spots. 
				\vspace{0.2cm}
			\end{flushleft}
		\end{minipage}
	
		\textbf{How:}\\
		Do not directly shine with light source on object, but rather indirectly with the help of a diffusive surface. It does not reduces the specular reflection, but increases the diffuse reflection component, yielding in less variations.
		\par 
		\textbf{Why:}\\
		Prevents sharp shadows and large intensity variations over glossy surface.  
	
		\subsubsection{Polarized-lighting}
		\subsubsection{Colored-lighting}
		\subsubsection{Structures-lighting}
		\subsubsection{Stroboscopic-lighting}
		
		\subsection{Cameras}
		
		\section{Feature Extraction}	
	
		 
		
	\end{multicols*}
	\setcounter{secnumdepth}{3}
\end{document}
