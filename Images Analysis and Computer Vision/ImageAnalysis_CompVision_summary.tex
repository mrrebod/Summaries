%
% Zusammenfassung Image Analysis & Computer Vision D-ITET
% ===========================================================================
% Author:			Marco Dober
% Version:			0.1
% Last changed: 	19.09.2019	
% ---------------------------------------------------------------------------

\documentclass[a4paper, fontsize=8pt, landscape, DIV=1]{scrartcl}
\usepackage{lastpage}
\usepackage{hyperref}
\usepackage[graphicx]{realboxes}
% Include general settings and customized commands
\input{settings/general}
\input{settings/commands}
%change page style for header
\pagestyle{fancy}
\footskip 20pt
\rhead{Marco Dober}
\lhead{Image Analysis \& Computer Vision}
\chead{\thepage}
\cfoot{}
\headheight 17pt \headsep 10pt
\title{Image Analysis \& Computer Vision}
\author{Marco Dober}
\date{\today}


\begin{document}
	\setcounter{secnumdepth}{3} %no enumeration of sections
	\begin{multicols*}{4}
		%
		\section*{Disclaimer}
		This summary is part of the lecture ``ETH Image Analysis \& Computer Vision'' by Prof. Van Gool, Prof. Konukoglu and Prof. Goksel  (HS19). It is based on the lecture. \\[6pt]
		Please report errors to \href{mailto:doberm@student.ethz.ch}{doberm@student.ethz.ch} such that others can benefit as well.\\[6pt]	
		The upstream repository can be found at \href{https://github.com/mrrebod/Summaries}{https://github.com/mrrebod/Summaries}
		\vfill\null
		\pagebreak
		
		\maketitle 
		\thispagestyle{fancy}
		
		\section{Introduction}
		Vision is important:
		\begin{itemize}[noitemsep, label={$\blacktriangleright$}]
			\item Half our brain is devoted to it
			\item Developed many times douring evolution
			\item It is non-contact
			\item It can be implemenmted with high-resolution
			\item Works with ambient EM-waves
			\item yields color, texture, depth, motion, shape
		\end{itemize}
		\textcolor{red}{Take home message}:\\
		\textbf{\textcolor{red}{For people vision is their most crucial sense, for good reason}}
		\subsection{Perception of vision}
		\begin{minipage}[t]{0.49\columnwidth}
			\begin{flushleft}
				{\centering Perception of \textbf{intensity}\\}
				\includegraphics[width=\columnwidth, height=2cm]{images/Introduction/perc_intensity.png} 
				The gray fields have the same intensity (same gray tone).
			\end{flushleft}
		\end{minipage}
		\begin{minipage}[t]{0.49\columnwidth}
			\begin{flushright}
				{\centering Perception of \textbf{color}\\}
				\includegraphics[width=\columnwidth, height=2cm]{images/Introduction/perc_color.png}
				The red squares have equal color
			\end{flushright}
		\end{minipage}
		\hrule
		\begin{minipage}[t]{0.49\columnwidth}
			\begin{flushleft}
				{\centering Perception of \textbf{length}\\}
				\includegraphics[width=\columnwidth, height=2cm]{images/Introduction/perc_length.png} 
				the horizontal lines are equally long.
			\end{flushleft}
		\end{minipage}
		\begin{minipage}[t]{0.49\columnwidth}
			\begin{flushright}
				{\centering \textbf{Lines being straight}
				\includegraphics[width=2cm, height=2cm]{images/Introduction/perc_straight.png}\\}
				The lines do not have any curvature
			\end{flushright}
		\end{minipage}	
		\hrule
		\begin{minipage}[t]{0.49\columnwidth}
			\begin{flushleft}
				{\centering Perception of \textbf{parallelism}\\}
				\includegraphics[width=\columnwidth, height=2cm]{images/Introduction/perc_parallel.png} 
				All lines are parallel.
			\end{flushleft}
		\end{minipage}
		\begin{minipage}[t]{0.49\columnwidth}
			\begin{flushright}
				{\centering Perception of \textbf{curvatures}
				\includegraphics[width=2cm, height=2cm]{images/Introduction/perc_curvature.png}\\}
				There is no spiral.
			\end{flushright}
		\end{minipage}
		\begin{minipage}[t]{0.49\columnwidth}
			\begin{flushleft}
				{\centering Perception of \textbf{motion}\\}
				\includegraphics[width=\columnwidth, height=3cm]{images/Introduction/perc_motion.png} 
				The pole rotates about the vertical, it does not translate vertically.
			\end{flushleft}
		\end{minipage}
		\begin{minipage}[t]{0.49\columnwidth}
		\begin{flushright}
				{\centering The role of \textbf{context}
				\includegraphics[width=3cm, height=3cm]{images/Introduction/perc_context.png}\\}
				All encircled patterns are identical!
			\end{flushright}
		\end{minipage}
		\par 
		\textcolor{red}{Take home message}:\\
		\textbf{\textcolor{red}{Effective vision needs more than sheer filtering and measuring}}
		\subsection{Applications}
		Most early applications where found in \textbf{production environments}, as these allow for \textbf{controlled conditions} and have \textbf{little uncertainty}. But some areas do not allow for much control: medical IP, remote sensing, surveillance, etc.\\
		Currently Computer Vision (CV) is conquering the less controllable areas by storm:
		\par  
		\begin{minipage}[t]{0.49\columnwidth}
			\begin{flushleft}
		 		{\centering \textbf{Image enhancement:} mobile $\rightarrow$ DSLR\\}
		 		\includegraphics[width=\columnwidth, height=5cm]{images/Introduction/app_enhancement.png} 
			\end{flushleft}
		\end{minipage}
		\begin{minipage}[t]{0.49\columnwidth}
			\begin{flushright}
				{\centering \textbf{Image retrieval, captioning}\\}
				\includegraphics[width=\columnwidth, height=5cm]{images/Introduction/app_retrieval.png}
			\end{flushright}
		\end{minipage}
		\hrule
		\begin{minipage}[t]{0.49\columnwidth}
			\begin{flushleft}
				{\centering \textbf{Autonomous vehicles}\\}
				\includegraphics[width=\columnwidth, height=3cm]{images/Introduction/app_autonomous_vehicles.png} 
			\end{flushleft}
		\end{minipage}
		\begin{minipage}[t]{0.49\columnwidth}
			\begin{flushright}
				{\centering \textbf{Visual surveillance}\\}
				\includegraphics[width=\columnwidth, height=3cm]{images/Introduction/app_surveillance.png}
			\end{flushright}
		\end{minipage}
		\hrule
		\begin{minipage}[t]{0.49\columnwidth}
			\begin{flushleft}
				{\centering \textbf{Augmented Reality, e.g sports}\\}
				\includegraphics[width=\columnwidth, height=3cm]{images/Introduction/app_augm_reality.png} 
			\end{flushleft}
		\end{minipage}
		\begin{minipage}[t]{0.49\columnwidth}
			\begin{flushright}
				{\centering \textbf{Computer-assisted surgery}\\}
				\includegraphics[width=\columnwidth, height=3cm]{images/Introduction/app_surgery.png}
			\end{flushright}
		\end{minipage}			
		\par 
		\textcolor{red}{Take home message}:\\
		\textbf{\textcolor{red}{It is feasible now to let most things see and interpret their environment.}}
		
		\subsection{The nature of light}
		\subsubsection{Light as an EM wave...}
		\subsubsection{Interactions with matter}
		Light as an EM wave...\\
		interaction with matter...\\
		
		
	\end{multicols*}
	\setcounter{secnumdepth}{3}
\end{document}
